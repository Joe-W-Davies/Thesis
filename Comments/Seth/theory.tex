There are a few changes to make so that you're consistent in presenting when mass terms 
are or aren't explicitly allowed in the Lagrangian, and why or why not. 
I think for bosons the explicit mass term spoils local gauge invariance, 
whereas for fermions the problem arises 
once you have the left/right structure that transforms differently.
So for example the QCD fermion mass term (eqn 1.11) is OK on its own 
because that theory by itself doesn't care about helicity, 
even though in the context of the SM that mass term actually has to come from the Higgs; 
you should be clearer about how this issue fits into your argument somewhere.
I think a bigger issue is your argument around eqn 1.19, 
which illustrates the problem with gauge invariance for fermions in the context of SU(2)xU(1) 
but then suddenly pivots to commenting on the photon.
You should also include what a photon mass term would look like and comment on the problem there.

At the bottom of p7 you could expand your discussion of the choice of parameterisation in eqn 1.26, 
and why exactly the photon is massless.
The point, at least the way I like to think about it, 
is that you have already defined the photon to be the massless particle that comes out of this; 
it's not something you prove using the equations for the W,Z, and photon on p5, 
but rather something you have made true by construction in the choices made there, 
which you're now implicitly "justifying" on p7.
Your text doesn't really specify whether this (and the relation to cos(thetaW)) 
is by construction or just a happy accident.

If it were me, I would move more of the details 
and motivation of the STXS categorization to the end of this chapter, 
but I don't feel strongly about it.

Minor point: in the middle of p5 you refer to "the U(1) boson", 
but at this point you haven't yet defined B_\mu 
so a pedantic reader might think you meant the A_\mu of electromagnetic theory. 
(Which you do mean mathematically but not physically.) 
I think just saying "a U(1) boson" is at least the fastest way to solve this.
