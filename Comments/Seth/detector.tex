%To keep in mind for later

p11:  How are 10 and 20 um resolutions achieved?  Are these for hits or tracks?  
For hits I would expect 100/sqrt(12) and 150/sqrt(12), if I'm remembering my formulae correctly.  
You can get another factor of 2 in the best case (certain overlap regions with 2 hit charge sharing), 
so then one gets 14 and 21 um, which I suppose are rounded to the nearest 10?   
Hmm, anyway, understanding my stream-of-consciousness here is a to-do item for your viva more than a point about the text, I suspect... 
This talk seems useful: [3]
[3] http://www.desy.de/~obehnke/stat/gean10/rohe.pdf

In general the level at which you discuss the reconstruction varies by subdetector 
(e.g. the ECAL calibration is really quite interdependent with the photon and electron algorithms).  
You might want to think a little more systematically about how your reconstruction discussion fits in, e.g. does it have its own section?  
Related: your explanation of Figure 3.8 in the text is quite cursory

You're using a mix of references, e.g. including conference proceedings. 
Eventually you should look back and use the most formal sources you can, which means papers > TDR's > conference proceedings, 
with of course the caveat that if you need specific facts and they're only in conference proceedings then you should keep them.


% I have questions / disagree

p6, bottom:  There's definitely more to the LHC than dipoles and quadroples [1], and quadroples are used for more than just focusing near IP's.  
Slides 15-16 in [2] seem useful.  
The point isn't that you need to say a lot more, but just to make sure your short summary is correct. %seems fine as-is to me...
[1] https://lhc-machine-outreach.web.cern.ch/lhc-machine-outreach/components/magnets/types_of_magnets.htm 
[2] https://www.slideshare.net/kumar_vic/lhc-construction-operation

p9: Eta definition: To avoid ambiguity, I recommend round brackets around \theta / 2, or else use \frac{\theta}{2}

p10: There's a (relatively brief) point missing from your initial description of the tracker: 
what are the material interactions with matter that determine which particles interact with the tracker, 
and what is the data collected for each track by each layer (as input to track reconstruction)?    
I see this is mentioned in the 2nd paragraph on a per-subdetector basis, but this seems too late to me.


% Done / agree
p7:  Analyses based upon this Run 1 dataset were able to discover the Higgs boson" --> May need a few citations

p8, bullet points: You should integrate the bold parts more clearly grammatically with the rest of the text. 
They should either be part of sentences, or else be separated with a colon or similar.  
Be consistent about whether these bullet points are sentence fragments or one or more complete sentences.

p8: "A cylindrical coordinate system is frequently to describe events within the CMS" --> proofread

p10: I think you have the tracker length and diameter flipped in the text

Fig 3.4: mention in the caption that the acronyms are spelled out in the text

p11, Last paragraph of 3.3.2: what "consequently" refers to is unclear gramatically (even if we can guess)

