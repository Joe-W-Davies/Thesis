\chapter{Statistical framework}
\label{chap:hgg_stats}

\section{Likelihoods and the extraction of results}
This chapter details how go from diphton invariant mass distribution to Higgs boson properties. Fixed mass.
In addition to the STXS measurements, other interpretations of the data are performed, including signal strength modifiers, both for inclusive Higgs boson production and for individual production modes, and coupling modifiers in the $\kappa$-framework~\cite{}.Categories are sensitive to! POIS.

Per category likelihood, total product of all.

\begin{equation}
    S^{\gamma\gamma}_{ij} = \sigma_i \cdot \mathcal{B}^{\gamma\gamma} \cdot \epsilon^{\gamma\gamma}_{ij} \cdot \mathcal{L}
\end{equation}

Total efficiency is the reconstruction and then category selection criteria. Effectively the probability of event of signal process, $i$, falling in category, $j$.

\section{Signal modelling}
\subsection{Shape}
Beamspot reweigh. 

\subsection{Normalisation}


\section{Background modelling}\label{sec:bkg_modeling}

\section{Systematic uncertainties}
A systematic uncertainty is included in the analysis to account for the minor differences between photon and electron. Signal shape etc. Vertex assignment.
\subsection{Theoretical uncertainties}

\subsection{Experimental uncertainties}

\subsection{Correlation schemes}