\chapter{Conclusions}
\label{chap:conclusions}


The standard model of particle physics provides an extremely successful description of the fundamental particles and forces, with its predictions surviving much experimental scrutiny. After the discovery of the Higgs boson at the LHC, the entire particle content of the SM has now been experimentally confirmed. Despite this, the SM fails to provide explanations for a number of observed phenomena, including neutrino masses, the presence of dark matter, and incorporation of the gravitational force. These shortcomings motivate theories beyond the SM, many of which modify the Higgs sector. A detailed characterisation of the Higgs boson is therefore a priority for particle physics experiments, such as those at the LHC.

The properties of the Higgs boson can be studied in a variety of ways, including through precision measurements in experimentally clean decay channels, or via searches for rare and forbidden interactions. This thesis presents a search for the rare decay of the Higgs boson to two electrons. The search is performed using pp collision data collected at $\sqrt{s}=13$~TeV by the CMS experiment at the LHC between 2016 and 2018, corresponding to an integrated luminosity of 138~fb$^{-1}$. Analysis categories are developed to target Higgs boson production via both gluon fusion and vector boson fusion. To improve the ratio of signal-to-background events, dedicated BDTs are trained for each Higgs boson production mode. These classifiers use characteristic features of the event topology to improve rejection of background, which consists primarily of Drell-Yan \Zee and \ttbar decays. The modelling of signal events by the BDT is also validated using a sample of \Zee events in a control region orthogonal to the selection defining the signal region. Since the analysis sensitivity is not close to the SM prediction, upper limits on the branching fraction for \Hee decays are extracted. Limits are determined via a maximum likelihood fit to the dielectron mass distribution in each analysis category. Models for signal events are derived using simulated samples at various Higgs boson masses, while background models are taken directly from a sideband region in data. The observed upper limit at the 95\% confidence level on the branching fraction for the \Hee decay is

\begin{equation*}
    \BHee < 3.0\times10^{-4}.
\end{equation*}

\noindent This limit is also translated into an upper bound on the Higgs boson coupling modifier to electrons, yielding $|\kappa_{e}|<240$. 

These results provide the most stringent constraints on the Higgs boson to electron pair decay channel to date. However, the observed limits remain orders of magnitude greater than the corresponding SM predictions. In order to improve the precision of existing measurements, including those in the \Hee channel, upgrades to the LHC, such as those discussed in Section~\ref{sec:HGCAL}, are crucial. In particular, the HL-LHC upgrade project is expected to collect over twenty times the data analysed in this thesis, improving the statistical precision of measurements made with the LHC Run 2 dataset by almost a factor of five. The upgrade will also facilitate entirely new measurements, such as a detailed characterisation of the Higgs boson potential. These benefits are, however, accompanied by several experimental challenges, which will require significant modifications to the existing detector technologies and reconstruction techniques, such as those developed in the CMS HGCAL project. Many of the proposed solutions are expected to rely heavily on the application of sophisticated ML algorithms, some of which have been discussed in chapter~\ref{chap:machineLearning}. Overall, however, the expected improvement in sensitivity offered by the HL-LHC is unfortunately not near to closing the five orders of magnitude difference between the observed limit on \BHee presented in this thesis, and the SM prediction. 

Beyond the LHC, future colliders offer a unique opportunity to probe the electron-Yukawa coupling. In particular, the Future Circular Collider~\cite{FCC_OG}, a proposed 100~km long successor to the LHC, is expected to collide electron and positron beams as part of its precision physics programme~\cite{FCC_physics_prog}. The FCC electron-positron collider will be operated at a selection of threshold energies for various electroweak processes, allowing for tens of attobarns of data to be collected. Dedicated runs at $\sqrt{s}=$ \mH would allow for resonant Higgs boson production via the electron fusion $s$-channel process, $\mathrm{ee}\rightarrow \mathrm{H}$, facilitating measurements of the electron-Yukawa coupling at the Higgs boson production vertex. Recent feasibility studies using 10~ab$^{-1}$ of integrated luminosity project a sensitivity to this coupling of 1.6 times the SM prediction~\cite{FCC_eeH}, providing unparalleled precision when compared with previous direct measurements. Future colliders are thus an exciting prospect with which to probe properties of the Higgs boson, and elucidate the nature of the Yukawa coupling to the first generation fermions. Through such tests, performed at the LHC and beyond, it is hoped that explanations for the shortcomings of the SM will be found, and our understanding of the universe, improved.

