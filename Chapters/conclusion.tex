\chapter{Conclusions}
\label{chap:conclusions}

The SM of particle physics has proved to be an extremely successful theory in describing the fundamental constituents of matter and their interactions via the strong, weak, and electromagnetic forces. Nevertheless, there are a number of physical observations that the SM does not account for including neutrino oscillations, dark matter, and most notably, gravity. This necessitates the existence of BSM physics. At high energy physics experiments, precision measurements offer an indirect approach to search for BSM physics, since as-yet-undetected new states can modify the predictions of the SM. In particular, precision measurements of Higgs boson properties will help to elucidate the origins of electroweak symmetry breaking, and perhaps point to an extension of the scalar sector in the SM.  This thesis reports the precision measurements of Higgs boson properties by the CMS experiment, using LHC Run 2 data.

Chapters~\ref{chap:hgg_overview}--\ref{chap:hgg_results} detail the measurements of Higgs boson production cross sections and couplings in the \Hgg decay channel. The results are based on 137~\fbinv of p-p collision data at $\sqrt{s}=13$~TeV, and are published in Ref.~\cite{Sirunyan:2021ybb}. Events with two reconstructed photons consistent with the decay of a Higgs boson are selected, and subsequently categorised to target different kinematic regions of the STXS framework. This is performed using a sophisticated chain of ML algorithms in order to maximise the sensitivity. A statistical inference procedure is applied to extract the Higgs boson properties from the diphoton invariant mass distribution in each analysis category. This involves the modelling of signal events as a peak around the Higgs boson mass, background events as a smoothly falling distribution, and accounting for the associated systematic uncertainties. The results are extracted using a maximum likelihood fit under different signal hypotheses, providing measurements of signal strength modifiers, coupling modifiers, and production cross sections in the STXS framework.

All observed measurements are consistent with the SM predictions within uncertainties. The inclusive Higgs boson signal strength, relative to the SM prediction, is measured to be $1.12 \pm 0.09$. Three measurements are performed within the STXS framework, in which 6, 17, and 27 independent kinematic regions are measured simultaneously. The latter demonstrates the most granular fit of Higgs boson production cross sections in a single decay channel to-date. Moreover, many of the kinematic regions are measured here for the first time, including the splitting of the ttH production mode into five different $p_T^H$ regions. Ultimately, this divide-and-measure approach of the STXS framework enhances the sensitivity to BSM physics which affects particular regions of the production phase space. One region of interest is ggH production with $p_T^H>200$~GeV, due to the sizeable enhancement that would arise from potential new physics states appearing in the ggH loop. The measured ggH cross section with $p_T^H>200$~GeV is compatible with the SM, with an observed value of $0.9^{+0.4}_{-0.3}$ relative to the SM prediction. Finally, an upper limit is placed on single-top associated Higgs boson production for the first time using \Hgg measurements at CMS. The observed (expected) limit at the 95\% confidence level is found to be 14 (8) times the SM prediction.

Chapter~\ref{chap:eft} reports the BSM interpretation of STXS measurements using an EFT approach, which has been made public by the CMS Collaboration in Ref.~\cite{CMS-PAS-HIG-19-005}. This approach benefits from being agnostic to the specifics of the BSM theory, such that the short-range UV physics is integrated out and modelled as effective contact interactions between the SM fields. In the interpretation, STXS measurements are combined across all major Higgs boson decay channels to provide the ultimate sensitivity and enable multiple EFT operators to be probed simultaneously. The Higgs Effective Lagrangian (HEL) is used to parametrise deviations in the Higgs boson cross sections and branching fractions as functions of the EFT Wilson coefficients. This parametrisation is applied in a maximum likelihood fit to extract constraints on seven independent EFT parameters. All results are compatible with the SM expectation, and the confidence intervals are amongst the tightest constraints placed on this subset of EFT operators, thereby reducing the parameter space for potential BSM physics.

The final chapter investigates the physics potential of the HL-LHC programme. It is quite remarkable to consider the huge advances that have been made in characterising the Higgs boson and its interactions, all within ten years of the particle's discovery. A prime example are the \Hgg results in this thesis, which demonstrate a precision of around 9\% on the inclusive Higgs boson production rate. Nevertheless, for a vast majority of measurements (e.g. in the STXS framework) the uncertainties are dominated by a lack of statistics; this means the accumulation of more high energy collision data is of paramount importance. At the HL-LHC, we will collect at least ten times the amount of data expected at the end of Run 3 of the LHC. In doing so, it becomes possible to constrain BSM physics that introduces per-cent level modifications to the Higgs boson couplings. Moreover, the increased data set facilitates new measurements, such as the indirect probe of the Higgs boson self coupling via top-associated differential cross section measurements, as shown in chapter~\ref{chap:hllhc}. The results demonstrate that additional sensitivity to the Higgs boson self coupling is available in single-Higgs measurements, which will become more apparent when combining with measurements from other Higgs boson production modes and decay channels. This analysis was published in a collection of HL-LHC projection studies in Ref.~\cite{Cepeda:2019klc}.

In summary, an extensive programme of work has been established at the LHC to characterise the Higgs boson. This thesis has presented a range of precision measurements of Higgs boson properties performed by the CMS experiment, where all measurements are found to be consistent with SM predictions. Despite this, there is increasing evidence from other areas of particle physics showing fundamental flaws in SM theory. With the Higgs boson lying at the centre of the SM, it is not unrealistic to assume that new BSM physics will interfere with the Higgs sector in some way. Looking to the future, it is therefore critical that we continue to improve the precision of Higgs boson measurements, both in terms of accumulating more data and using more sophisticated analysis techniques. In doing so, we will achieve a more fundamental understanding of our universe.