\chapter{Conclusions}
\label{chap:conclusions}
%FIXME consider adding something quantitative here? Or in the results summary section?

Measurements of Higgs boson production cross sections are performed using \SI{77.4}{\fbinv} of 
data at $\sqrt{s} = \SI{13}{TeV}$ collected by the CMS experiment at the LHC.
Events with two signal-like photons consistent with the decay of the SM Higgs boson are selected
and subsequently categorised using kinematic variables, the diphoton BDT, and the dijet BDT.
Models of the signal and background contributions to the diphoton invariant mass distribution 
in each analysis category are constructed as inputs to the final fits
from which the results of the analysis are extracted.
The results are presented as various measurements of cross sections within the STXS framework;
the best-fit values and their uncertainties are estimated with an approach 
based on the profile likelihood ratio test statistic.

The observed measurements are all consistent with the hypothesis of a SM Higgs boson.
Cross sections for gluon fusion and vector boson fusion production, 
normalised to the corresponding standard model predictions,
are measured to be $1.15_{-0.15}^{+0.15}$ and $0.8_{-0.31}^{+0.37}$ respectively~\cite{HIG-18-029}.
Two additional measurements are performed with a mixture of merged and unmerged 
stage 1 STXS bins as the parameters of interest.
In the first case, seven signal parameters are defined, 
each with a measured uncertainty of less than 100\% of the cross section predicted by the SM.
The second fit has thirteen signal parameters, all of which are measured to a precision 
of better than 200\% of the SM prediction.
The results of both fits are compatible with the SM; the $p$-values with respect to the SM hypothesis
are found to be approximately 64\% and 18\% respectively.

Despite the excellent agreement with the SM, 
many of the measured uncertainties are still relatively large.
This is particularly true for the stage 1 cross section measurements, 
for which the uncertainties are currently dominated by the statistical component.
Plausible BSM theories often predict that Higgs coupling parameters 
deviate from the SM at the per-cent level~\cite{Snowmass}.
The STXS framework is well-suited to systematically characterising 
any possible deviations from the SM, for example within effective field theories~\cite{STXStoEFT}.
Furthermore, 
in the near future combinations of analyses using the full LHC Run 2 dataset will be performed.
Inputs will include results from the various decay channels and from different experiments, 
resulting in significant improvements in precision on stage 1 cross section measurements.
In many cases, it is likely that the magnitude of the uncertainties will be comparable to both 
the errors on the SM theoretical predictions 
and the experimental systematic uncertainties on the measurements.
Further progress will require advancement in experimental techniques 
and understanding of collected data, 
as well as dedicated efforts to improve the accuracy of SM predictions.

In the longer term, Run 3 of the LHC and subsequently the Phase 2 upgrade to the HL-LHC 
will provide unprecedented amounts of data for analysis.
There are significant experimental challenges to be met in order for these data 
to remain of the same quality as those collected during Run~2.
One aspect of this is the HGCAL, which brings exciting possibilities for novel reconstruction
techniques that have only begun to be explored.
In addition to facilitating improved precision on existing measurements 
in the Higgs sector~\cite{FutureYR}, the HL-LHC will enable entirely new measurements to be made.
For example, the nature of the Higgs potential has not yet been confirmed experimentally.
This can be measured directly by searching for events where two Higgs bosons are produced, 
but can also be accessed indirectly, 
for example by precisely measuring differential distributions 
of single Higgs boson production~\cite{JonnoPAS}.
A multitude of other further measurements will also be possible, 
including observing rare Higgs boson decay modes.

In summary, there is huge potential for further progress in Higgs boson measurements at the LHC.
Performing optimal measurements will require a sustained and dedicated experimental effort
to understand the data and reduce systematic uncertainties, 
as well as adopting new and innovative analysis techniques.
Progress on the theoretical will also be necessary 
to keep the uncertainties on the SM predictions below the experimental precision.
In doing so, the SM will be tested as thoroughly as possible.
Hopefully, insights into how to address its shortcomings will be found,
and our fundamental understanding of the universe improved.
