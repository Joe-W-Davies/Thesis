\chapter{Signal and background modelling}
\label{chap:sigbkg}

\section{Introduction}

The final results of this analysis are extracted by performing a maximum likelihood fit 
of the signal and and background models to the diphoton invariant mass distribution observed in data.
Models of the signal and background \mgg distributions are therefore required as inputs to the fit.
The signal model is derived from simulation, 
with a model constructed for each particle level stage 1 bin in each reconstructed analysis category.
Both the shape and normalisation of the model are parameterised as a function of \mH.
The data-driven background model considers a range of different functional forms to 
represent the smoothly falling background scectrum, 
following the approach described in Ref~\cite{Envelope}.
The construction of both the signal and background models is described in detail in this chapter.
In addition, the treatment of systematatic uncertainties affecting the two models is discussed here.

\section{Signal modelling}

\section{Background modelling}

\section{Systematic uncertainties}
