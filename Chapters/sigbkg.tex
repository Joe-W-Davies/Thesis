\chapter{Signal and background modelling}
\label{chap:sigbkg}

\section{Introduction}

The final results of this analysis are extracted by performing a maximum likelihood fit 
of the signal and and background models to the diphoton invariant mass distribution observed in data.
Models of the signal and background \mgg distributions are therefore required as inputs to the fit.
The signal model is derived from simulation, 
with a model constructed for each particle level stage 1 bin in each reconstructed analysis category.
Both the shape and normalisation of the model are parameterised as a function of \mH.
The data-driven background model considers a range of different functional forms to 
represent the smoothly falling background scectrum, 
following the approach described in Ref~\cite{Envelope}.
The construction of both the signal and background models is described in detail in this chapter.
In addition, the treatment of systematatic uncertainties affecting the two models is discussed.

\section{Signal modelling}

The signal model is a parametric function of \mH which describes the shape of the \mgg distribution, 
together with the expected normalisation of this shape.
An independent model is constructed for each stage 1 bin in each analysis category
Additionally, since the \mgg shape depends on whether the right vertex (RV) 
or wrong vertex (WV) has been chosen, the model for each of these cases is constructed separately.

Each model consists of a sum of up to five Gaussian functions.
The number of Gaussian functions required depends on the shape of the \mgg spectrum 
and the available MC statistics.
Alternative functional parameterisations have been studied, 
such as the sum of a Gaussian function to represent the core of the \mgg distribution 
and an exponential to model each tail \cite{LouieThesis}.
This alternative parameterisation yields very similar signal models to the sum of Gaussian functions, 
and the final results are unaffected.

The parameters of the Gaussian functions are determined by performing a fit 
to the simulated \mgg distribution for each model.
In order to account for the fact that the mass of the Higgs boson is not known exactly, 
the model constructed is a continuous parametric function of \mH.
The dependence on \mH is determined by simultaneously fitting events simulated with 
three values of \mH: 120, 125, and \SI{130}{GeV}.
Each parameter of each Gaussian function is represented as a linear function of \mH; 
the shape of each model then consists of $2\left(3N_{\textrm{Gaus}}-1\right)$ parameters, 
where $N_{\textrm{Gaus}}$ is the chosen number of Gaussian functions.
The values of these parameters are all established by the simultaneous fit across mass points.

%TODO insert interpolation figure here

In some cases, there is an insufficient number of simulated events available for a given stage 1 bin, 
analysis category and vertex scenario combination. %this is set at 200
The shape is then replaced by that of the stage 1 bin 
which has the highest expected yield in the category under consideration.
This replacement procedure is motivated by the fact that events subject to the same selection
tend to have similar values of the diphoton mass resolution.

After each model has been constructed, the shapes from the RV and WV scenarios are combined.
The fraction of events in which the correct vertex is chosen 
is also described by a linear function of \mH; 
once determined, this is used to assign the correct normalisations for the RV and WV models.

%TODO insert proc x cat signal model figure here

For each category, 
the models corresponding to the contributions from each stage 1 bin are then summed.
To normalise the contribution from each stage 1 bin correctly, 
the total number of expected events for each stage 0 process is obtained 
using the cross-sections and \Hgg branching ratio from Ref.~\cite{YR4}, 
\footnote{The exception is ggH, 
for which the latest calculations at N3LO are used \cite{Anastasiou2016,Anastasiou2017}.}
and the measured intergrated luminosity in data.
The fraction of each stage 1 bin is then taken from the simulated events for each stage 0 process.
Finally, the product of the detector efficiency and analysis acceptance 
is modelled as a linear function of \mH, 
determined by the ratio of the total number of expected events 
to the number of events entering each analysis category.
Together these allow the total signal model for each category to be computed.

%TODO insert cat signal model figure here

%TODO insert sum of cats

\section{Background modelling}

\section{Systematic uncertainties}
