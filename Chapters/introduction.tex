\chapter{Introduction}
\label{chap:intro}

The Standard Model (SM) of particle physics is a theory which describes
the fundamental structure of matter and its interactions.
Guided by the results of experiments studying the behaviour of high-energy particles, 
the SM unifies the electromagnetic and weak forces and places them in a coherent framework
together with the strong force.
The SM has been extraordinarily successful, 
with many of its predictions verified to unprecedented precision.
A key aspect of the SM is the Higgs mechanism, 
which breaks the symmetry of the electroweak interaction, explains how particles obtain mass, 
and predicts the existence of a fundamental particle known as the Higgs boson.
The Higgs boson was observed experimentally by the ATLAS and CMS Collaborations in 2012
with data collected during Run 1 of the Large Hadron Collider (LHC)~\cite{ATLASdiscovery,CMSdiscovery}.
This discovery completed the particle content and modern understanding of the SM, 
and represents another great success of the theory.

The SM is however known to be incomplete as a theory of nature.
Firstly, it does not include a description of the force of gravity.
The SM also does not provide a suitable candidate for dark matter, 
which is required to explain certain astrophysical phenomena~\cite{BulletCluster},
and forms a key part of the modern understanding 
of the large scale structure of the universe~\cite{Planck}.
Furthermore, in the SM neutrinos are treated as massless, 
whereas experimental measurements of neutrino oscillations confirm that they are not~\cite{NeutrinoOscillation}.
A wide range of extensions to the SM have been proposed 
which provide explanations for some or all of these phenomena~\cite{SUSY}.
These beyond standard model (BSM) theories can predict the existence of entirely new particles, 
and thereby modify the predictions made by the SM.

The results presented in this thesis test the SM predictions of Higgs boson production
in the diphoton decay channel with the CMS experiment.
Data collected as part of Run~2 of the LHC during 2016 and 2017 are analysed, 
yielding a dataset of \SI{77.4}{\fbinv}.
The CMS detector is a multi-purpose apparatus able to reconstruct several types of high-energy particles.
Its design is motivated in part by the objective of precisely measuring the properties of the Higgs boson.
In particular, the performance of its electromagnetic calorimeter is excellent, 
which enables effective reconstruction of the photons arising from the decay of the Higgs 
and is vital for the success of the \Hgg analysis.
The final results of the analysis are measurements of Higgs boson production cross sections
within the simplified template cross section (STXS) framework~\cite{YR4}.
The STXS framework provides a coherent approach to performing measurements of Higgs boson couplings, 
minimising the effect of theoretical uncertainties 
whilst simultaneously permitting the use of advanced experimental techniques.
Results within the STXS framework are presented
which aim to test the SM prediction as stringently as possible.
Deviations from the SM predictions may indicate the presence of BSM physics 
and guide the way to an improved understanding of nature, 
or otherwise tightly constrain the possible forms BSM theories can take.

This thesis presents results of the CMS \Hgg analysis using 2016 and 2017 data, 
which is documented in Ref.~\cite{HIG-18-029}.
The structure of the thesis is as follows.

Chapter~\ref{chap:theory} describes the structure of the SM as a gauge field theory, 
before introducing the Higgs mechanism and its implications.
The phenomenology of the Higgs boson and its decay into two photons is discussed, 
and the latest measurements of Higgs boson properties, including those within the STXS framework, 
are summarised.

Chapter~\ref{chap:detector} describes the structure of the LHC and of the CMS detector itself.
The role of each subdetector and how they capture the various products 
of high energy proton-proton collisions is explained.

Chapter~\ref{chap:hgcal} describes the planned upgrade program for the LHC to become the High-Luminosity LHC (HL-LHC) 
and the required changes to the CMS detector, focusing on the High Granularity Calorimeter (HGCAL).
The motivation for and design of the HGCAL is presented, 
together with studies illustrating its performance in object reconstruction and physics analysis.

Chapter~\ref{chap:objects} describes how events recorded by the CMS detector are reconstructed.
Emphasis is placed on the procedures used to form photon objects, 
and how they are subsequently combined with a vertex hypothesis to form diphoton candidates 
for the \Hgg analysis.

Chapter~\ref{chap:categorisation} describes the procedure for categorising events in the \Hgg analysis.
Analysis categories targeting different production processes are defined, 
and the sensitivity of the analysis improved 
using machine learning techniques to discriminate between signal and background processes.

Chapter~\ref{chap:sigbkg} describes the construction of models for the signal and background 
contributions to the diphoton invariant mass distributions of the final analysis categories.
The treatment of systematic uncertainties in the analysis is also detailed.

Chapter~\ref{chap:results} describes the statistical fitting procedure used 
to extract the final results of the analysis and their associated uncertainties.
Results of cross sections at different levels of granularity within the STXS framework 
are presented, and comparisons made to the corresponding SM predictions.

Finally, Chapter~\ref{chap:conclusions} discusses the conclusions drawn from the results of the analysis, 
and provides a perspective on future work to build on those results.
