\chapter{Introduction}
\label{chap:intro}

% \textbf{To do}: update introduction to match revised structure of thesis. Pull this

% For millennia, we have sought for a deeper understanding of our universe. That said, it was not until the birth of quantum physics in the 20$^{\rm{th}}$ century that we began to accurately describe the elementary constituents of our universe and the fundamental laws that govern them. From the beginning of the quantum revolution to the present day, huge advancements in both theory and experiment have led to a fruitful, discovery-rich era for particle physics. Nevertheless, a number of fundamental questions are left unanswered. In our aim for a complete theory describing the laws of the universe, it is therefore necessary to continue probing the fundamental structure of nature.

Our current best theory for describing elementary particles and their interactions is known as the Standard Model (SM) of particle physics~\cite{Glashow:1961tr,Weinberg:1967tq,Salam:1968rm}. 
%Anchored by the notion of symmetry,
The SM successfully describes the strong force as well as the unification of the weak and electromagnetic forces. Central to this so-called electroweak theory is the Brout-Englert-Higgs (BEH) mechanism~\cite{Englert:1964et,HIGGS1964132,Higgs:1964pj,Guralnik:1964eu,PhysRev.145.1156,PhysRev.155.1554}, responsible for the spontaneous symmetry breaking of the electroweak interaction and the subsequent generation of mass for other elementary particles. Additionally, the BEH mechanism predicts the existence of a scalar boson with mass situated around the electroweak scale; namely the Higgs boson. In 2012, the Higgs boson was observed experimentally~\cite{Aad:2012tfa,Chatrchyan:2012xdj,Chatrchyan:2013lba} by the ATLAS and CMS Collaborations~\cite{Aad:2008zzm,Chatrchyan:2008zzk} at the Large Hadron Collider (LHC)~\cite{1748-0221-3-08-S08001}, CERN. This discovery was heralded as a great triumph of the SM, marking the completion of the particle content of the theory.

Despite all its glory, the SM is known to be incomplete. For one, the SM fails to explain the fourth fundamental force of nature: gravity. 
%Despite huge efforts in the field, we appear to have reached an impasse in developing a fully quantised theory of Einstein's general relativity~\cite{}. 
Secondly, cosmological models which successfully describe the large scale structure of the universe predict the existence of dark matter and dark energy~\cite{Aghanim:2018eyx}, which hold no place in SM theory. The SM also fails to explain the fine-tuning of the Higgs field vacuum expectation value, known as the hierarchy problem~\cite{PhysRevD.13.974,PhysRevD.20.2619}, and it does not explain the fact that neutrinos have mass, as required by the experimental observation of neutrino oscillations~\cite{Fukuda:1998mi}. On top of this, the SM lacks a mathematical elegance by requiring a relatively large number of input parameters, whose values are not predicted by the theory and appear to be somewhat arbitrary, often provided by experiment~\cite{Zyla:2020zbs}. It is these shortcomings of the theory which make necessary the existence of new physics beyond-the-Standard Model (BSM).

At the LHC, there are two complementary methods used to search for BSM physics. The direct approach aims to explicitly observe new particles in data. Alas, since the discovery of the Higgs boson in 2012 there has been no direct evidence of new particles, suggesting that any BSM physics lies beyond the energy reach of the collider, or that it has a too-small cross section and therefore remains (so far) undetected. As a result, attention has shifted towards the second approach: indirectly probing new physics via precision measurements. Short-range interactions with BSM particles may leave a measurable imprint on the properties of SM particles. Hence, quantities well predicted in SM theory offer a unique tool for discovery, where deviations between measured and predicted values provide an indication of new physics. 

Precision measurements of Higgs boson properties will provide a better understanding of electroweak symmetry breaking, help to elucidate the nature of the Yukawa interactions with fermions, and even shed light on the origins of the universe via the shape of the Higgs potential~\cite{Kajantie:1995kf,Csikor:1998eu}. Furthermore, being the only fundamental scalar in the SM, the Higgs boson lies at the heart of many proposed BSM theories, such as supersymmetry, composite models or extra dimensions~\cite{Martin:1997ns,Witzel:2019jbe,Quiros:2013yaa}. This has led to the development of a broad and comprehensive program of work to characterise the Higgs boson and measure its couplings to other particles. Since discovery, the ATLAS and CMS Collaborations have observed all the major Higgs boson production modes, as well as the couplings of the Higgs boson to the third generation quarks and charged lepton~\cite{Aaboud:2018urx,Aaboud:2018zhk,Aaboud:2018pen,Sirunyan:2018hoz,Sirunyan:2018kst,Sirunyan:2017khh}. Most recently, the CMS experiment reported the first direct evidence of the Higgs boson coupling to the muon~\cite{Sirunyan:2020two}. Moreover, both experiments have performed differential measurements of Higgs boson properties to further scrutinise SM theory in specific regions of the Higgs boson phase space~\cite{ATLAS:2020wny,ATLAS-CONF-2019-029,Aad:2020jym,Sirunyan:2020hwz,Sirunyan:2020tzo}. So far all measurements are consistent with the SM predictions.

This thesis details precision measurements of Higgs boson properties using proton-proton collision data collected by the CMS experiment during Run 2 of the LHC. Firstly, Chapter~\ref{chap:theory} provides the theoretical foundations on which the measurements reside. Chapter~\ref{chap:cms} then describes the CMS experiment at the LHC, focusing on the design choices which enable Higgs boson precision measurements. The following three chapters provide a complete description of the analysis documented in Ref.~\cite{Sirunyan:2021ybb}, which measures Higgs boson production cross sections and couplings in the diphoton decay channel~(\Hgg). This channel is particularly powerful due to its clean final state topology, where the excellent energy resolution of the CMS electromagnetic calorimeter leads to a narrow peak in the diphoton invariant mass spectrum, effectively distinguishing Higgs boson production from SM background processes. Furthermore, it is one of the few channels which has reasonable sensitivity to all of the principal Higgs boson production modes. 

The \Hgg analysis is configured to perform measurements within the simplified template cross section (STXS) framework~\cite{deFlorian:2016spz}, which offers a coherent approach to Higgs boson production cross section measurements with increasing statistics. In the framework, the inclusive Higgs boson production phase space is divided into kinematic regions, which are split first by the production mode and subsequently by the kinematics of the event constituents. By measuring the cross section in these bins, a more complete description of Higgs boson production is achieved. The three chapters dedicated to this analysis are structured as follows. In Chapter \ref{chap:hgg_overview}, the event reconstruction and categorisation are described. Following this, Chapter \ref{chap:hgg_stats} explains the statistical inference techniques which are used to extract the Higgs boson cross sections and couplings. This includes the construction of the likelihood function, the modelling of both signal and background events, and the treatment of systematic uncertainties. Finally, Chapter \ref{chap:hgg_results} presents the results of the analysis in terms of signal-strengths, coupling-modifiers and production cross sections in the STXS framework.

Following on from this, Chapter~\ref{chap:eft} describes a BSM interpretation of STXS measurements using effective field theory (EFT)~\cite{BUCHMULLER1986621,Hagiwara:1993qt,Giudice_2007,Grzadkowski:2010es,Contino:2013kra}. In the EFT, BSM particles are postulated to have masses at an energy scale far beyond the accessible energy scale at the LHC. By performing a series expansion of the SM Lagrangian, we encapsulate all the information of a BSM ultraviolet (UV) complete theory, in terms of the infrared (IR) SM fields. This provides an (almost) model independent framework on which to probe BSM physics. Firstly, cross sections in the STXS framework and the Higgs boson branching fractions are expressed in terms of the EFT parameters. This parametrisation is then applied to a combination of STXS measurements performed by the CMS experiment in different Higgs boson decay channels~\cite{CMS-PAS-HIG-19-005}. Combining measurements in this way provides the optimal precision with regards to the current data set, and enables a number of directions in the EFT parameter space to be probed simultaneously. The results are shown as constraints on a set of EFT parameters, thereby reducing the potential parameter space for BSM physics.

The thesis concludes with a look to the future operation of the LHC machine, known as the High-Luminosity LHC (HL-LHC). Here, protons will be collided at five times the LHC design luminosity, providing a wealth of proton-proton collision data on which to base precision measurements. In Chapter~\ref{chap:hllhc}, a machine learning algorithm trained to discriminate photons and electrons from hadronic activity in the future CMS trigger system is described~\cite{CERN-LHCC-2020-004}, showing the expected performance of the algorithm in the HL-LHC environment. Following this, the future sensitivity to Higgs boson measurements at the HL-LHC is discussed. This features a projection study investigating an indirect method for probing the Higgs boson self-interaction via (single) Higgs boson production in association with top-quarks~\cite{CMS-PAS-FTR-18-020}. Finally, Chapter~\ref{chap:conclusions} draws conclusions from the results of this thesis, and offers a perspective on the future of Higgs boson precision measurements at the LHC and beyond.