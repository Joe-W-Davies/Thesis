\chapter{Introduction}
\label{chap:intro}

For millennia, we as humans, have sought for a deeper understanding of our universe. That said, it was not until the birth of quantum physics in the 20$^{\rm{th}}$ century that we began to accurately describe the elementary constituents of our universe and the fundamental laws that govern them. From the beginning of the quantum revolution to the present day, huge advancements in both theory and experiment have led to a fruitful, discovery-rich era for particle physics. Nevertheless, a number of fundamental questions are left unanswered. In our attempt for a complete theory describing the laws of the universe, it is therefore necessary to continue probing the fundamental structure of nature.

Our current best theory for describing elementary particles and their interactions is known as the standard model (SM) of particle physics~\cite{Glashow:1961tr,Weinberg:1967tq,Salam:1968rm}. Anchored by the notion of symmetry, the SM successfully describes the strong force as well as the unification of the weak and electromagnetic forces. Central to this so-called electroweak theory is the Higgs-Brout-Englert (HEB) mechanism~\cite{Englert:1964et,HIGGS1964132,Higgs:1964pj,Guralnik:1964eu,PhysRev.145.1156,PhysRev.155.1554}, responsible for the spontaneous symmetry breaking of the electroweak interaction and the subsequent generation of other elementary particle's mass. Additionally, the HEB mechanism predicts the existence of a scalar boson with mass situated around the electroweak scale; namely the Higgs boson. In 2012 the ATLAS and CMS experiments~\cite{Aad:2008zzm,Chatrchyan:2008zzk} at the Large Hadron Collider (LHC)~\cite{1748-0221-3-08-S08001}, CERN, confirmed the observation of the Higgs boson~\cite{Aad:2012tfa,Chatrchyan:2012xdj,Chatrchyan:2013lba}. This discovery was heralded as a great triumph of the SM, marking the completion of the particle content of the theory.

Despite all it's glory, the SM is known to be incomplete. For one, the SM fails to explain the fourth fundamental force of nature: gravity. Despite huge efforts in the field, we appear to have reached an impasse in developing a fully quantised theory of Einstein's general relativity~\cite{}. Cosmological models which successfully describe the large scale structure of the universe predict the existence of dark matter and dark energy~\cite{Aghanim:2018eyx}, which again hold no place in SM theory. The SM also fails to explain the fine-tuning of the Higgs field vacuum expectation value, known as the hierarchy problem~\cite{PhysRevD.13.974,PhysRevD.20.2619}, nor does it explain the fact that neutrinos have mass, confirmed experimentally by the observation of neutrino oscillations~\cite{Fukuda:1998mi}. On top of this, the SM lacks a mathematical elegance by requiring as input a number of free parameters, whose values are not predicted by the theory and appear to be somewhat arbitrary, often provided by experiment~\cite{Zyla:2020zbs}. It is these so-called shortcomings of the theory which make necessary the existence of new physics beyond-the-standard model (BSM).

At the LHC, there are two complementary methods used to search for BSM physics. The direct approach aims to explicitly observe new particles in data. Alas, since the discovery of the Higgs boson in 2012 there has been no direct evidence of new particles, suggesting that any BSM physics lies beyond the energy reach of the collider. As a result, attention has shifted towards the second approach: indirectly probing new physics via precision measurements. Short-range interactions with BSM particles may leave a measurable imprint on the properties of SM particles. Hence, quantities well predicted in SM theory offer a unique tool for discovery, where deviations between measured and predicted values provide an indication of new physics. 

By precisely measuring the properties of the Higgs boson we are able to better understand electroweak symmetry breaking, elucidate the nature of Yukawa interactions with fermions and even shed light on the origins of the universe via the shape of the Higgs potential~\cite{Kajantie:1995kf,Csikor:1998eu}. Furthermore, being the only fundamental scalar in the SM, the Higgs boson lies at the heart of many proposed BSM theories, such as supersymmetry, composite models or extra dimensions~\cite{Martin:1997ns,Witzel:2019jbe,Quiros:2013yaa}. This has led to the development of a broad and comprehensive program of work to characterise the Higgs boson and measure it's couplings to other particles. Since discovery, the ATLAS and CMS collaborations have observed all the major Higgs boson production modes, as well as the couplings of the Higgs boson to the third generation fermions~\cite{Aaboud:2018urx,Aaboud:2018zhk,Aaboud:2018pen,Sirunyan:2018hoz,Sirunyan:2018kst,Sirunyan:2017khh}. Most recently, the CMS experiment reported the first direct evidence of the Higgs boson coupling to the muon~\cite{Sirunyan:2020two}. Moreover, both experiments have performed differential measurements of Higgs boson properties to further scrutinise the SM theory in specific regions of the Higgs boson phase space~\cite{ATLAS:2020wny,ATLAS-CONF-2019-029,Aad:2020jym,Sirunyan:2020hwz,Sirunyan:2020tzo}. So far all measurements are consistent, within uncertainties, with the SM predictions.

This thesis details precision measurements of Higgs boson properties using proton-proton collision data collected by the CMS experiment during Run 2 of the LHC. The first half focuses on the measurement of Higgs boson properties in the diphoton decay channel~(\Hgg)~\cite{CMS-PAS-HIG-19-015}. This channel is particularly useful due to its clean final state topology, where the excellent photon energy resolution of the CMS electromagnetic calorimeter leads to a narrow peak in the diphoton invariant mass spectrum, effectively distinguishing Higgs boson production from SM background processes. Furthermore, the decent branching fraction of approximately 0.23\% allows for reasonable statistics in all of the major Higgs boson production modes. The analysis is configured to perform measurements within the simplified template cross section (STXS) framework~\cite{deFlorian:2016spz}: a coherent approach to Higgs boson production cross section measurements with increasing statistics. In the framework the inclusive Higgs boson production phase space is divided into discrete regions or ``bins", which are split firstly by production mode and subsequently by the kinematics of the event constituents. Hence, by measuring the cross section in these bins, a highly granular description of Higgs boson production is achieved.

The second half of the thesis describes an interpretation of Higgs boson measurements using effective field theory (EFT)~\cite{BUCHMULLER1986621,Hagiwara:1993qt,Giudice_2007,Grzadkowski2010,Contino:2013kra}. In the EFT, BSM particles are postulated to exist at an energy far beyond the accessible scale at the LHC. By performing a series expansion of the SM Lagrangian, we encapsulate all the information of a BSM ultraviolet (UV) complete theory, in terms of the infrared fields i.e. the SM fields. This provides an (almost) model independent framework on which to probe BSM physics, where deviations from SM predictions can be expressed in terms of the higher order operators of the EFT. Firstly, cross sections in the STXS framework and the Higgs boson branching fractions are expressed in terms of the parameters of both the Higgs Effective Lagrangian~(HEL)~\cite{Alloul:2013naa} and the SM Effective Field Theory~(SMEFT)~\cite{Brivio:2017vri}. STXS measurements performed by the CMS experiment, are then combined across decay channels in order to extract constraints on a number of EFT parameters, thereby constraining the phase space in which BSM physics could be hiding.

The results presented are from the CMS \Hgg analysis using the full Run 2 data and the latest CMS Higgs combination, documented in Refs.~\cite{CMS-PAS-HIG-19-015}~and~\cite{CMS-PAS-HIG-19-005} respectively. The Higgs combination has been improved by including the latest \Hgg result and the SMEFT interpretation. The structure of the thesis is as follows: Chapter {\color{blue}\ref{chap:theory}} builds a theoretical foundation which will aid in the understanding of all future chapters. Firstly, the Higgs mechanism is introduced in the context of the SM. This leads to a discussion of Higgs boson phenomenology at the LHC and an overview of the latest Higgs boson property measurements. After then describing the STXS framework in detail, the chapter closes with an introduction to EFT and it's implications with respect to Higgs boson physics.

In Chapter {\color{blue}\ref{chap:cms}}, the LHC and the CMS detector are described, focusing on the design choices which enable Higgs boson property measurements. Furthermore, a planned upgrade of the CMS machine for the High-Luminosity phase of the LHC (HL-LHC)~\cite{Contardo:2020886,ApollinariG.:2017ojx} is discussed; in particular the High-Granularity Calorimeter (HGCAL) project which will replace the current CMS endcap calorimeters~\cite{CERN-LHCC-2017-023}. An algorithm which distinguishes electrons and photons from hadronic activity in the HGCAL is presented.

Chapters {\color{blue}\ref{chap:hgg_overview}}, {\color{blue}\ref{chap:hgg_stats}} and {\color{blue}\ref{chap:hgg_results}} cover the CMS \Hgg analysis. In Chapter {\color{blue}\ref{chap:hgg_overview}}, the event reconstruction and categorisation is described. Following this, Chapter {\color{blue}\ref{chap:hgg_stats}} details the statistical framework on which the measurements are performed, where the modelling of both signal and background events and the treatment of systematic uncertainties are explained. Chapter {\color{blue}\ref{chap:hgg_results}} presents the results of the analysis in terms of signal strengths, coupling modifiers and production cross sections at stage 1.2 of the STXS framework.

Chapter {\color{blue}\ref{chap:eft}} details the methodology for constraining EFT parameters using STXS measurements. This includes the derivation of the cross section and branching ratio parametrisations using the HEL and SMEFT models. Following this, the results of the CMS Higgs combination in terms of the EFT parameters are presented and discussed.

In Chapter {\color{blue}\ref{chap:acceptance_studies}}, an extension to the EFT parametrisation is explored which includes the acceptance effects of the CMS detector. Building on from this, Chapter {\color{blue}\ref{chap:eft_future}} briefly summarises the future prospects and potential improvements for EFT measurements at CMS.

Somewhat standalone, Chapter {\color{blue}\ref{chap:self_coupling}} discusses an indirect probe of the Higgs boson self-coupling using single Higgs boson production measurements in the top associated production modes. The potential impact of performing this analysis in the diphoton decay channel at the HL-LHC is presented~\cite{Cepeda:2019klc,CMS-PAS-FTR-18-020}.

Finally, Chapter {\color{blue}\ref{chap:conclusions}} provides a conclusion of all results presented in this thesis, and offers an outlook on the future of Higgs boson property measurements at CMS and beyond.