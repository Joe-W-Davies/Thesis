\chapter{Event Reconstruction and Selection}
\label{chap:objects}

\section{Introduction}

The CMS \Hgg analysis is able to perform measurements of Higgs boson properties using the diphoton invariant mass distribution.
Photon pairs resulting from Higgs decays produce a narrow signal peak, centred at the value of the Higgs mass, in this distribution.
This can be measured on top of the smoothly falling background distribution, produced by other SM processes.
The Higgs mass is inferred from two photons by constructing the diphoton invariant mass, which is given by the following expression
\begin{equation}
\mgg = \sqrt{ 2 E_{\gamma_1} E_{\gamma_2} (1 - \cos{\theta}) }
\end{equation}
where $E_{\gamma_1}$ and $E_{\gamma_2}$ are the energies of each photons, and $\theta$ is the opening angle between them.
The sensitivity of the analysis is maximised when the diphoton mass resolution is reconstructed as precisely as possible.
This requires the two photons to be correctly identified and their positions and energies accurately measured.
Furthermore, the location of the interaction vertex from which the photons originated must be established in order to calculate the opening angle.
Additional objects in the event, including jets and leptons, are further used to improve the sensitivity of the analysis and measure different Higgs production processes.

This section describes the CMS procedure for reconstructing physics objects using the particle flow algorithm.
In addition, the photon and vertex identification techniques specific to the \Hgg analysis are detailed.

\section{Particle flow}

\section{Samples}
\subsection{Data}
\subsection{Simulation}

\section{Photon reconstruction}
\subsection{Photon energy}
\subsection{Photon identification}
\subsection{Photon pre-selection}

\section{Vertex reconstruction}
\subsection{Vertex selection}
\subsection{Vertex probability}

\section{Jet reconstruction}

\section{Reconstruction of other objects}
\subsection{Electons}
\subsection{Muons}
\subsection{Taus}
\subsection{Missing energy}
