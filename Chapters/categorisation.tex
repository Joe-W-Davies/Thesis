\chapter{Event Categorisation}
\label{chap:categorisation}

\section{Introduction}

The sensitivity of the \Hgg analysis depends on the ability to distinguish the narrow signal peak 
from the smoothly falling background in the diphoton mass distribution.
Events passing the preselection described in Chapter~\ref{chap:objects} 
are therefore subject to further categorisation, in order to increase 
the ratio of the number of signal events to the number of background events (S/B).
This enhances the sensitivity of the analysis, 
meaning the expected uncertainties on the measured quantities are reduced.

Analysis categories are also constructed to target events in which the Higgs boson was 
produced by a specific production mechanism. 
This is achieved using the information provided by additional objects in the event, 
alongside the two photons arising from the Higgs boson decay.
As well as facilitating measurements of cross sections corresponding 
to individual production mechanisms, these dedicated categories also enable the S/B to be improved.

In the previous \Hgg analysis using the 2016 dataset \cite{HIG-16-040}, 
dedicated categories targeting the VBF, ttH, and VH modes were construced, 
with the remaining so-called ``Untagged" categories were composed of mostly ggH events.
Here a similar approach is employed, 
but with additional divisions targeting individual stage 1 bins for the ggH and VBF processes.
Events selected by the 2017 \Hgg analysis targeting ttH production, 
described in Ref.~\cite{HIG-18-018}, are not included in this analysis. 
This is achieved by first applying the same selection criteria used to construct the ttH categories, 
and then removing them from further consideration. 
There are no dedicated analysis categories for VH production.

The categorisation targeting ggH is based on the reconstructed diphoton transverse momentum (\ptgg) 
and the number of jets in the event. 
A BDT referred to as the diphoton BDT is then used to reduce the amount of background. 
The VBF analysis categories make use of the same diphoton BDT 
to reduce the number of background events. 
Additionally, a BDT targeting the kinematics of the characteristic VBF dijet system, 
known as the dijet BDT, is utilised to reduce the contamination from ggH events.

Due to the conditions differing between the two years, 
the analysis is optimised separately for the 2016 and 2017 datasets. 
Once the analysis categories for each dataset are defined, simultaneous fits are
performed to the categories from both years to measure the chosen parameters of interest. 

\section{Gluon fusion categorisation}
\subsection{Signal bin definitions}
\subsection{Diphoton BDT}
\subsection{Category definitions}

\section{Vector boson fusion categorisation}
\subsection{Signal bin definitions}
\subsection{Dijet BDT}
\subsection{Category definitions}
