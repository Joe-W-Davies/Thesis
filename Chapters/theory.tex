\chapter{Theory}
\label{chap:theory}

\epigraph{\textit{Dreaming a thought that could ~\\
                  dream about a thought ~\\
                  That could think of the ~\\
                  dreamer that thought ~\\
                  That could think of dreaming and ~\\
                  getting a glimmer of God}
                 }{Frank Ocean}

\section{Introduction}
The standard model (SM) of particle physics is the current best description of nature 
at its most fundamental level.
Developed in the 1970s~\cite{Glashow,Weinberg,Salam}, 
the SM incorporates the electromagnetic, weak, and strong forces in a single coherent framework, 
unifying the electromagnetic and weak interactions in doing so.
The SM is a type of quantum field theory known as a gauge theory, 
which represents the fundamental constituents of matter and the forces between them
as excitations of relativistic quantum fields.
Many of its predictions have been experimentally verified to unprecedented precision~\cite{EMstuff}.
The spontaneous breaking of gauge symmetry in the unified electroweak sector of the SM
results in the prediction of the Higgs boson, 
the existence of which is now experimentally confirmed~\cite{HiggsShit}.
In this chapter, the fundamental particles and forces making up the SM are described, 
before its structure in terms of two gauge fields is explained.
The origin of the Higgs boson and through spontaneous symmetry breaking is elucidated.
The phenomenology of the Higgs boson and the consequences for experimental measurements 
are then detailed. 
Lastly, the simplified template cross section (STXS) framework is introduced 
and the latest precision measurements of the Higgs boson's properties are summarised.

\section{Fundamental particles and forces}
%go through the fundamental particles and forces of nature
In the SM, particles particles other than the Higgs boson are divided into spin-half fermions
and spin-one bosons.
Fermions are the fundamental constrituents of matter, 
and can either interact only via the electroweak force (leptons), 
or via both the electroweak and strong forces (quarks).
Both leptons and quarks exist in three distinct generations;
the three particles comprising each group have identical properties 
except for mass, which increases across generations.
This structure is shown in Table~\ref{tab:theory_fermions}, 
which shows the charge and mass of each type of fermion in the SM.
In addition, each particle has a corresponding antiparticle, 
which have the same mass but whose charge has the opposite sign to the original particle.
%Leptons are exist as free particles but quarks are confined to composite objects 
%such as the proton due to the nature of the strong force.

The interactions between fermions are mediated by a second class of fundamental particles, 
the gauge bosons.
Three forces are represented by the SM gauge bosons: 
the electromagnetic force, the weak nuclear force, and the strong force.
The mediator of the electromagnetic force is the massless photon, 
whilst the weak interaction is occurs via the exchange of three massive particles, 
the $W^+$, $W^-$, and $Z$ bosons.
Due to the unification of these two forces into the electroweak sector of the SM, 
the photon, $W^\pm$, and $Z$ bosons arise from combinations of the fundamental gauge fields.
The gluons, which mediate the strong force, 
are the only bosons in the SM which carry the charge the force's corresponding charge.
Consequently they interactions between gluons is possible.
The structure of the gauge bosons in the SM is summarised in Table~\ref{tab:theory_bosons}.

The final particle in the SM is the Higgs boson, 
which is the only scalar (spin-zero) particle in the theory.
The Higgs boson is a massive particle which arises 
due to spontaneous symmetry breaking in the electroweak sector, 
and whose interactions explain the masses of the other SM particles.

\section{Gauge fields}
%general intro to gauge fields, E-L, Dirac, Nother etc

\subsection{Electroweak interactions}
%SM structure in EW sector

\subsection{Strong interactions}
%SM structure in QCD

\section{Spontaneous symmetry breaking}
%the mechanism, and how it gives mass to W and Z
%and fermion masses
%and then link to the prediction of the Higgs boson itself and how it interacts

\section{Properties of the Higgs boson}
%production, decay etc

\section{The simplified template cross section framework}
%introduction and motivation
%stage 0 bin definitions
%state of the art measurements, i.e. HIG-16-040
%stage 1 bin definitions
