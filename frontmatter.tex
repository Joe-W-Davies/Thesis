\chapter*{\centering Abstract}

A search is presented for the Higgs boson decay to a pair of electrons in proton-proton collisions at a centre-of-mass energy of 13~TeV. The data set was collected with the Compact Muon Solenoid experiment at the Large Hadron Collider between 2016 and 2018, corresponding to an integrated luminosity of 138~\fbinv. The analysis develops event categories targeting Higgs boson production via gluon fusion and vector boson fusion, defined by selection on dedicated machine learning-based classifiers. An upper limit on the Higgs boson branching fraction to an electron pair is determined as $3.0\times10^{-4}$ at the 95\% confidence level, which is the most sensitive limit to date.

\chapter*{\centering }% Dedication}
\begin{center}
    \thispagestyle{empty}
    To Mum, Dad,\\ and my sister, Liv
\end{center}


\chapter*{\centering Declaration}
I, the author, declare that all work presented in this thesis is my own. The studies have been conducted in collaboration with several individuals, as members of the Compact Muon Solenoid (CMS) Collaboration. All figures produced by the author, or in collaboration with the author, are labelled as ``CMS" if taken directly from a CMS publication, ``CMS Preliminary" if taken from a CMS preliminary public document, and figures which have not explicitly been made public by the CMS Collaboration attain the label ``CMS Work in Progress", or are simply left unlabelled. The figures and studies that are taken from external sources, and are not produced directly by the author, or in collaboration with the author, are appropriately referenced throughout.


\begin{flushright}
    Joseph William Davies
\end{flushright}
%%%%%%%%%%%%%%%%%%%%%%%%%%%%%%%%%%%%%%%%%%%%%%%%%%%%%%%%%%%%%%%%%%%%%%%%%%%%%%%%%%%%%%%%%%%%%%%%%%%%%%%%%%%%%%%%%%%%%%%%%%%%%%%%%
\chapter*{\centering Acknowledgements}


There are so many people who have helped make the last four years an enjoyable and fulfilling experience. This thesis has been greatly shaped by their influence, and although I cannot name each and every person here, I am extremely grateful to you all.

Firstly, I would like to thank the HEP department at Imperial College for allowing me to study my PhD here. The group is full of warm and witty personalities, each with their own valuable expertise to share. I could not think of a better place to study, and will miss it greatly when I leave. Thank you also to the STFC for funding my work, without which none of this would have been possible.

Nick, thank you for your guidance throughout my studies. Despite your many commitments, you have always found time to answer my (often naive) questions, and these conversations have proved invaluable. 
Ed, you have supported me throughout almost the entirety of my PhD, all the way from TwoStep, to submission of this thesis. I've learned so much from your mentorship, I think both academically and otherwise, and will always carry these lessons with me. 
Jon, you were one of the first people to convince me to study at Imperial. For that, and for so much more throughout the last few years, I am sincerely grateful. It has been a real privilege to work with you and Ed, and to say that this thesis would not have been possible without you both is an understatement.

To my other friends at Imperial, both present and past, thank you for keeping me sane and always being up for a laugh. My time here was made so much more enjoyable by your company. Thank you also to my friends at home who have provided humour, support, and beverages, at much needed times.

To my family --- thank you for everything. Your love has been an enduring constant in my life, and for that I am forever grateful. Dad, although physics is perhaps/absolutely not your own interest, your support throughout my studies has allowed me to pursue it as mine. You are calm, patient, caring, and I could not ask for a better role model. Mum, I could have filled this entire page writing about what you have done for me and Liv. Your love and compassion is the bedrock of my accomplishments, and thus, this thesis is your achievement as much as it is mine. Susie, I cannot tell you how important your support for me has been over the last four years. Were it not for your patience and encouragement, there would be no thesis to write. I am so grateful to have you in my life. Finally, Liv --- I feel incredibly lucky to have grown up with you and shared almost all of our lives together. You're my best friend and I'm so proud of everything you do. It is one of the great privileges of my life to call you my twin. 

I love you all very much, and promise not to make you read all of this.





%%%%%%%%%%%%%%%%%%%%%%%%%%%%%%%%%%%%%%%%%%%%%%%%%%%%%%%%%%%%%%%%%%%%%%%%%%%%%%%%%%%%%%%%%%%%%%%%%%%%%%%%%%%%%%%%%%%%%%%%%%%%%%%%%


\tableofcontents
\listoffigures
\listoftables

\chapter*{\centering Preface}

This thesis presents a complete description of the search for Higgs boson decays to two electrons published in Ref~\cite{HIG-21-015-PAS}. I am responsible for all aspects of the analysis workflow, from the development of event categorisation procedures to extraction of the final results. In addition to this search, I made significant contributions to the \Hgg Simplified Template Cross Section analysis, published in Ref~\cite{HIG-19-015}. For brevity, this work will not be discussed in this thesis. A summary and relevant accreditation for each chapter is given below.

\textbf{Chapter 1} introduces this work, with reference to the current landscape of Higgs boson coupling measurements.

\textbf{Chapter 2} describes the theory underpinning the standard model of particle physics, with emphasis on the Brout-Englert-Higgs mechanism. This is entirely the work of others, summarised in my own words.

\textbf{Chapter 3} provides an overview of key concepts in the field of machine learning, pertinent to later chapters of this thesis. This is the work of others summarised in my own words.

\textbf{Chapter 4} briefly describes the Large Hadron Collider, before focusing on the Compact Muon Solenoid detector, which was designed, built, and operated by others. This chapter concludes with studies of photon and electron identification techniques in the Level-1 Trigger for the Phase-2 upgrade to the CMS calorimeter endcaps. These studies include a baseline description of the models used, which were developed by myself and Jonathon Langford, as well as memory and latency optimisation for model deployment, which is my own work.

\textbf{Chapter 5} describes the techniques used to reconstruct particles in the CMS detector, which is the work of others, although the validation of the electron energy scale corrections for the \Hee analysis was checked by myself. This chapter also outlines the simulated samples and data used to perform the \Hee search.

\textbf{Chapter 6} describes the categorisation of \Hee events to target the gluon fusion and vector boson fusion Higgs boson production mechanisms. The content presented in this chapter is entirely my own work, including the construction, interpretation, and validation of Boosted Decision Tree classifiers used in the nominal approach, and studies into deep learning techniques for categorisation of VBF events.

\textbf{Chapter 7} describes the development of signal and background models, as well as the systematic uncertainties included in this analysis, and potential bias within the envelope method. This work is my own, although the original techniques, which were modified by myself for this analysis, were developed by others. 

\textbf{Chapter 8} reports the results of this analysis. This is my own work, although the methodology and tools were developed by others.

\textbf{Chapter 9} gives a summary of this analysis in my own words.

\chapter*{}
\epigraph{
  \textit{``One day I will find the right words, and they will be simple."}}
          {Jack Kerouac}
\cleardoublepage

