\chapter*{\centering Abstract}
Measurements of Higgs boson production cross sections 
with the Higgs boson decaying into a pair of photons are reported.
Events with two photons are selected from a sample of proton-proton collisions 
at a centre-of-mass energy of \SI{13}{TeV} 
collected by the Compact Muon Solenoid detector at the Large Hadron Collider in 2016 and 2017,
corresponding to a total integrated luminosity of $77.4~\mathrm{fb}^{-1}$.
Cross sections for gluon fusion and vector boson fusion production, 
relative to the corresponding standard model predictions,
are measured to be $1.15 \pm 0.15$ and $0.83_{-0.31}^{+0.37}$ respectively.
These two production modes are further measured in kinematic regions 
within the simplified template cross section framework.
All results are found to be in agreement with the standard model expectations.


\chapter*{\centering }% Dedication}
\begin{center}
    \thispagestyle{empty}
    To Mum and Dad
\end{center}


\chapter*{\centering Declaration}
The work contained within this thesis is my own. 
It was produced using existing work from, and in collaboration with, 
several individuals and the Compact Muon Solenoid (CMS) Collaboration. 
The details of the contributions relevant to each chapter are set out below;
in all cases, the work of others has been referenced appropriately.

Chapter~\ref{chap:intro} introduces our current understanding of particle physics 
and the Higgs boson in my own words.

Chapter~\ref{chap:theory} explains the theory of the Standard Model and the Higgs mechanism, 
which is the work of others, in my own words. 
The status of the latest Higgs boson measurements is summarised, 
which includes my own work on the CMS \Hgg analysis based on the 2016 dataset~\cite{HIG-16-040}.

Chapter~\ref{chap:detector} describes the CMS detector, 
which was designed, built, and operated by others, in my own words.

Chapter~\ref{chap:hgcal} describes the High Granularity Calorimeter (HGCAL), 
the design and testing of which is the work of others within the CMS collaboration. 
The section on reconstruction is based on my own work, 
which was performed together with Lindsey Gray, Clemens Lange and Emilio Meschi~\cite{ClusteringConf}.
The demonstration of the HGCAL's potential for a vector boson fusion analysis 
in the \Hgg decay channel is my own work~\cite{HGCAL}.

Chapters~\ref{chap:objects}-\ref{chap:results} describe the methodology and results of the CMS \Hgg analysis
which constitute the main part of this thesis and consist primarily of my own work.
This analysis has also been documented in Ref.~\cite{HIG-18-029}.
The detailed breakdown of each chapter is given below.

Chapter~\ref{chap:objects} describes how events are reconstructed at CMS.
These techniques were developed and optimised by others within the \Hgg analysis group 
and the CMS collaboration, but are summarised in my own words.

Chapter~\ref{chap:categorisation} describes the categorisation of events 
to maximise the sensitivity of the analysis.
This is my own work, and is based on the strategies adopted in previous CMS \Hgg analyses.
The data-driven method for the training of the dijet boosted decision tree 
was initially developed by Yacine Haddad, and was implemented in this analysis by Shameena Bonomally.

Chapter~\ref{chap:sigbkg} describes the construction of the signal and background models, 
and the various uncertainties included in the analysis.
This is my own work, although the techniques utilised were first developed by others.

Chapter~\ref{chap:results} reports the final observed results and their uncertainties.
This is also my own work, but the methods and tools used are originally the work of others.

Chapter~\ref{chap:conclusions} summarises the work done in this thesis 
and the implications of the results.
This is in my own words and with my own considerations for the future development of this work.

\begin{flushright}
    Edward John Titman Scott
\end{flushright}


\chapter*{\centering Acknowledgements}
Many, many people have helped me through the past three and a half years, 
far more than I can list here.
I am sincerely grateful to all of you; I have felt very lucky throughout my time at Imperial.

Firstly, thank you to the Imperial HEP group and STFC for together 
allowing me to do this PhD in the first place. 
There is a great environment both here and at CERN, 
and the opportunity to spend time out in Geneva was a fantastic one.
The various conferences and travel opportunities have been terrific too.

This thesis contains the contributions of hundreds of members of the CMS Collaboration.
Working as part of CMS has meant a lot to me, 
and the \Hgg group in particular was lively, warm, and, at times, hilarious. 
I would like to thank the various CMS \Hgg conveners for helping get us through 
the numerous approval processes over the years.

The IC \Hgg has been a brilliant place to work, starting with my two supervisors. 
My experience as a student has been immeasurably improved by the kindness of both of you.
Gavin, I cannot thank you enough for your incredible wisdom and patience; 
you have always kept me on track and focused on the bigger picture.
Seth, you are the main reason this thesis was possible at all. 
The amount of help you have given me over the years is astonishing, 
as is your dedication. 
The humourous spirit with which you handled all that has kept me both sane and happy, somehow.
I'd also like to thank Chris Seez for the help out at CERN, 
always delivered in a very much no-nonsense manner; I greatly enjoyed my HGCAL work because of you.
Also thanks to Nick Wardle, for answering many stupid questions and always being up for a laugh.

To my friends at IC and from CERN: thank you so much. 
We have had a pretty intense time, but together it's been fun.
I'm proud of us all, in advance.
Thank you also to my friends at home, who mean so much to me.
Your loyalty, support, laughter and provision of all-round decent times have been invaluable.

Mum and Dad: thank you for everything. 
Your love and support have been the most important things in my life.
In my memory at least, this came more often in the form of driving me to and from Croydon 
for basketball practices than particularly encouraging me to wonder about physics, 
but it had the same effect.
You always encouraged me to pursue anything that I wanted to pursue, 
and made me believe it was worth doing so.
I cannot thank you enough for that.
To Jonny and Adam; I would not be who I am without you two.
Growing up together has been more than anyone could ask for, 
and I amazed and proud of who you have become.

Finally, Ari. 
Spending the last ten (!) years with you has been the best thing I have ever done.
Your patience, support, laughter, and passion mean the world to me. 
I love you more than anything.


\tableofcontents
\listoffigures
\listoftables


\chapter*{}
\epigraph{
  \textit{``And once you have tasted flight, \\ \\
            you will walk the Earth \\ \\
            with your eyes turned skyward; \\ \\
            for there you have been, \\ \\
            and there you long to return''}}
         {Leonardo da Vinci}


\cleardoublepage
