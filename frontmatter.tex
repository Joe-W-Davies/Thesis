\chapter*{\centering Abstract}
Measurements of Higgs boson production cross sections 
with the Higgs boson decaying into a pair of photons are reported.
Events with two photons are selected from a sample of proton-proton collisions 
at a centre-of-mass energy of \SI{13}{TeV} 
collected by the Compact Muon Solenoid detector at the Large Hadron Collider in 2016 and 2017,
corresponding to a total integrated luminosity of $77.4~\mathrm{fb}^{-1}$.
Cross sections for gluon fusion and vector boson fusion production, 
normalised to the corresponding standard model predictions,
are measured to be $1.15 \pm 0.15$ and $0.83_{-0.31}^{+0.37}$ respectively.
These two production modes are further measured in kinematic regions 
within the simplified template cross section framework.
All results are found to be in agreement with the standard model expectations~\cite{HIG-18-029}.


\chapter*{\centering }% Dedication}
\begin{center}
    \thispagestyle{empty}
    To Mum and Dad
\end{center}


\chapter*{\centering Declaration}
The work contained within this thesis is my own. 
It was produced using existing work from, and in collaboration with, 
several individuals and the Compact Muon Solenoid (CMS) Collaboration. 
The details of the contributions relevant to each chapter are set out below;
in all cases, the work of others has been referenced appropriately.

Chapter~\ref{chap:intro} introduces our current understanding of particle physics 
and the Higgs boson in my own words.

Chapter~\ref{chap:theory} explains the theory of the Standard Model and the Higgs mechanism, 
which is the work of others, in my own words. 
The status of the latest Higgs boson measurements is summarised, 
which includes my own work on the CMS \Hgg analysis based on the 2016 dataset~\cite{HIG-16-040}.

Chapter~\ref{chap:detector} describes the CMS detector, 
which was designed, built, and operated by others, in my own words.

Chapter~\ref{chap:hgcal} describes the High Granularity Calorimeter (HGCAL), 
the design and testing of which is the work of others within the CMS collaboration. 
The section on reconstruction is based on my own work, 
which was performed together with Lindsey Gray, Clemens Lange and Emilio Meschi~\cite{ClusteringConf}.
The demonstration of the HGCAL's potential for a vector boson fusion analysis 
in the \Hgg decay channel is my own work~\cite{HGCAL}.

Chapters~\ref{chap:objects}-\ref{chap:results} describe the methodology and results of the CMS \Hgg analysis
which constitute the main part of this thesis and consist primarily of my own work.
This analysis has also been documented in Ref.~\cite{HIG-18-029}.
The detailed breakdown of each chapter is given below.

Chapter~\ref{chap:objects} describes how events are reconstructed at CMS.
These techniques were developed and optimised by others within the \Hgg analysis group 
and the CMS collaboration, but are summarised in my own words.

Chapter~\ref{chap:categorisation} describes the categorisation of events 
to maximise the sensitivity of the analysis.
This is my own work, and is based on the strategies adopted in previous CMS \Hgg analyses.
The data-driven method for the training of the dijet boosted decision tree 
was initially developed by Yacine Haddad, and was implemented in this analysis by Shameena Bonomally.

Chapter~\ref{chap:sigbkg} describes the construction of the signal and background models, 
and the various uncertainties included in the analysis.
This is my own work, although the techniques utilised were first developed by others.

Chapter~\ref{chap:results} reports the final observed results and their uncertainties.
This is also my own work, but the methods and tools used are originally the work of others.

Chapter~\ref{chap:conclusions} summarises the work done in this thesis 
and the implications of the results.
This is done in my own words and with my own considerations for the future development of this work.

\begin{flushright}
    Edward John Titman Scott
\end{flushright}


\chapter*{\centering Acknowledgements}
Acknowledgements go here.
%STFC & CERN
%CMS Hgg group, especially the conveners?
%IC Hgg group; names?
%Seth
%Gavin
%Chris
%Nick
%Friends at IC and at CERN: keeping sanity, getting a beer, and decent magret de canard
%Friends at home: loyalty, support, laughter, etc
%M and D; thank you for everything. 
%This was more often in the form of driving to Croydon for basketball practices
%than encouraging me to wonder about physics, but it had the same effect.
%You always encouraged me to pursue anything that I wanted to pursue, 
%and made me believe it was worth doing so.
%Ari


\tableofcontents
\listoffigures
\listoftables


%\chapter*{}
%\epigraph{
%  \textit{``And once you have tasted flight, \\ \\
%            you will walk the Earth \\ \\
%            with your eyes turned skyward; \\ \\
%            for there you have been, \\ \\
%            and there you long to return''}}
%         {Leonardo da Vinci}


\cleardoublepage
